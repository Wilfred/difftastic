\documentclass{article}
\usepackage[utf8]{inputenc}
\usepackage{hyperref}

\date{\today}
\author{Eric Förster \and Patrick Förster}
\title{\TeX{}Lab}

\begin{document}

\maketitle{}

\section{Introduction}

TexLab is a cross-platform implementation of the
\href{https://microsoft.github.io/language-server-protocol/specifications/specification-current/}{Language Server Protocol}
for the \LaTeX{} typesetting system.
It aims to produce high quality code completion results.
The server may be used with any editor that implements the Language Server Protocol.
It is written in Rust, a blazingly fast systems programming language.

\section{Features}

The language server implements most of the Language Server Protocol specification.
In addition to that, it implements additional functionality like
building and forward search.

\section{Availability}

TexLab is available on \href{https://github.com/latex-lsp/texlab}{GitHub},
various package managers and CTAN\@. 
Pre-compiled binaries are available on the 
\href{https://github.com/latex-lsp/texlab/releases}{GitHub Releases} page.
Some editor extensions are able to automatically download TexLab.

\section{Installation}

There are various ways to install TexLab:
\begin{itemize}
    \item 
        TexLab is included in some package managers like \texttt{brew},
        \texttt{pacman} and \texttt{scoop}.
        Please refer to the badges in the README to see if your package manager
        includes TexLab.
    \item
        You can download a pre-compiled binary from our 
        \href{https://github.com/latex-lsp/texlab/releases}{GitHub Releases} page.
    \item
        Some extensions like the Visual Studio Code extension or
        \texttt{coc-texlab} can automatically download the server for you.
    \item
        You can download the sources from either GitHub or CTAN
        and compile the server with \texttt{cargo build --release}.
        The \texttt{texlab} binary can be found inside \texttt{target/release}.    
\end{itemize}

\section{Usage}

\subsection{Synopsis}

\texttt{texlab [FLAGS] [OPTIONS]}

\subsection{Flags}

\begin{itemize}
    \item \texttt{-h}, \texttt{--help} Prints help information
    \item \texttt{-q}, \texttt{--quiet} No output printed to stderr
    \item \texttt{-V}, \texttt{--version} Prints version information
    \item \texttt{-v}, \texttt{--verbosity} Increase message verbosity (\texttt{-vvvv} for max verbosity)
\end{itemize}

\subsection{Options}

\begin{itemize}
    \item \texttt{--log-file <FILE>} WRite the logging output to \texttt{FILE}
\end{itemize}

\end{document}